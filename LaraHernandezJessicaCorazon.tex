\documentclass{article}
\textheight=19cm
\textwidth=14cm
\topmargin=2cm
\usepackage{amsmath,amssymb,amsfonts,latexsym}
\usepackage[utf8]{inputenc}
\begin{document}
難しい感情だった
始めは人を恨んだ
それから不幸を祈っていた
許されるだなんて思っちゃいない

恥ずかしい毒も吐いた
続けて人を誹謗した
どこから自分を失っていた
嘲笑われるくらいに腐っていた

後を悔やむことがないのなら
後で悔やむと書きはしないと
不甲斐ない航海の旅路を綴っている

どうか!! どうしたってなれない 夢ばっか選んで
どうにだってならない 嘘なんかついて
買い被った完全な 沈没船を救ってよ
どうか!! もう終わってしまったんだって生命を投げ捨て
もう嫌だって頬を伝った遭難信号に
気づいて 合図した
CQCQ 聞こえますか

(間奏)

叶わない感情だった
受けるべき天罰なんだ
それくらい人を蹴落としてきて
同情のひとつさえも欠いていた
ひとり声に出してしまった
あなた以外は もうどうなってもいい
それ以上の声を押し込んでいた
胸の奥が張り裂けてしまうから

(間奏)

明日は明日で上書きできると
今日をドブに捨てた今日でした
荒れ狂う逆風に 未来は吹き飛んだ

こうして! どう足掻いて前向いたって 夢は遠ざかった
どうもがいて 帆を張ったって嘘にしか見えなくて
ユートピアと命名した幽霊船は沈んでく

そうして!! もう終わってしまったんだって生命を投げ捨て
もう嫌だって頬を伝った遭難信号に
気づいて アウトプットした本当の自分
ふいに ついに消えかけた
CQCQ 聴こえますか
\end{document}
